%% ------------------------------------------------------------------------- %%
\chapter{Gradient Boosting Machines}
\label{cap:boosting-intro}

\textit{Gradient Boosting} is a machine learning technique, based on the idea of additive models from statistics. An additive model is a  regression technique suggested by \sloppy{\cite{doi:10.1080/01621459.1981.10477729}}, where the basic idea is to approximate a dataset using a sum of \textbf{smooth functions} of the individual features of the observed data, instead of a complex general regression surface.

\section{Additive Model}\index{additive!model}\index{regression}

A supervised dataset is a set of pairs $(x^{(i)}, y^{(i)})$, where $x^{(i)}$ denotes the $p$-dimensional vector of the $i$-th observation ($p$ is the number of features), and $y^{(i)}$ is the true outcome of the $i$-th observation. An additive model is then described as:
$$E[y^{(i)}\mid x^{(i)}] = \beta_0 + \sum_{j=1}^pf_j(x^{(i)}_j)$$

The $\beta_0$ is the intercept term, and each function $f_j$ is learned from the respective $j$-th feature over all the observations in the dataset. Each of those functions can be estimated using any nonparametric regression technique, such as Gaussian Process Regression, smoothing splines, kernel regression, k-nearest-neighbors, etc.

\section{Gradient Boosting}\index{gradient!boosting}\index{boosting}

The Gradient Boosting algorithm



Para a escrita de textos em Ciência da Computação, o livro de Justin Zobel, 
\emph{Writing for Computer Science} \cite{zobel:04} é uma leitura obrigatória. 
O livro \emph{Metodologia de Pesquisa para Ciência da Computação} de 
Raul Sidnei Wazlawick \cite{waz:09} também merece uma boa lida.
Já para a área de Matemática, dois livros recomendados são o de Nicholas Higham,
\emph{Handbook of Writing for Mathematical Sciences} \cite{Higham:98} e o do criador
do \TeX, Donald Knuth, juntamente com Tracy Larrabee e Paul Roberts, 
\emph{Mathematical Writing} \cite{Knuth:96}.

O uso desnecessário de termos em lingua estrangeira deve ser evitado. No entanto,
quando isso for necessário, os termos devem aparecer \emph{em itálico}.

\begin{small}
\begin{verbatim}
Modos de citação:
indesejável: [AF83] introduziu o algoritmo ótimo.
indesejável: (Andrew e Foster, 1983) introduziram o algoritmo ótimo.
certo : Andrew e Foster introduziram o algoritmo ótimo [AF83].
certo : Andrew e Foster introduziram o algoritmo ótimo (Andrew e Foster, 1983).
certo : Andrew e Foster (1983) introduziram o algoritmo ótimo.
\end{verbatim}
\end{small}

Uma prática recomendável na escrita de textos é descrever as legendas das
figuras e tabelas em forma auto-contida: as legendas devem ser razoavelmente
completas, de modo que o leitor possa entender a figura sem ler o texto onde a
figura ou tabela é citada.  

Apresentar os resultados de forma simples, clara e completa é uma tarefa que
requer inspiração. Nesse sentido, o livro de Edward Tufte \cite{tufte01:visualDisplay},
\emph{The Visual Display of Quantitative Information}, serve de ajuda na 
criação de figuras que permitam entender e interpretar dados/resultados de forma
eficiente.

% \emph{Thesis are random access. Do NOT feel obliged to read a thesis from beginning to end.}



%% ------------------------------------------------------------------------- %%
\section{Considerações Preliminares}
\label{sec:consideracoes_preliminares}

Considerações preliminares\footnote{Nota de rodapé (não abuse).}\index{genoma!projetos}.
% index permite acrescentar um item no indice remissivo
Texto texto texto texto texto texto texto texto texto texto texto texto texto
texto texto texto texto texto texto texto texto texto texto texto texto texto
texto texto texto texto texto texto texto.
 

%% ------------------------------------------------------------------------- %%
\section{Objetivos}
\label{sec:objetivo}

Texto texto texto texto texto texto texto texto texto texto texto texto texto
texto texto texto texto texto texto texto texto texto texto texto texto texto
texto texto texto texto texto texto.

%% ------------------------------------------------------------------------- %%
\section{Contribuições}
\label{sec:contribucoes}

As principais contribuições deste trabalho são as seguintes:

\begin{itemize}
  \item Item 1. Texto texto texto texto texto texto texto texto texto texto
  texto texto texto texto texto texto texto texto texto texto.

  \item Item 2. Texto texto texto texto texto texto texto texto texto texto
  texto texto texto texto texto texto texto texto texto texto.

\end{itemize}

%% ------------------------------------------------------------------------- %%
\section{Organização do Trabalho}
\label{sec:organizacao_trabalho}

No Capítulo~\ref{cap:boosting-intro}, apresentamos os conceitos ... Finalmente, no
Capítulo~\ref{cap:boosting-intro} discutimos algumas conclusões obtidas neste
trabalho. Analisamos as vantagens e desvantagens do método proposto ... 

As sequências testadas no trabalho estão disponíveis no Apêndice \ref{ape:sequencias}.
